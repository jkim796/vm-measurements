\section{Introduction}
\label{sec:intro}
%
Operating Systems (OS) is a piece of software that manages and virtualizes hardware for applications.
%
Operating Systems (OS) is a piece of software that manages and virtualizes hardware for applications.
%
As OS abstracts away the complexity of orchestrating various hardware components such as CPUs, Memory, and I/O as well as different applications, it has been instrumental in reducing the burden of the programmers.
%
Virtualization refers to adding another level of indirection to run OSes on an abstraction of hardware such as CPUs, GPUs, and FPGAs.
%
However, the virtualization can not only be categorized by the target platform it virtualizes but also at which level it virtualizes the system.
%
For example, the 1st Generation Virtual Machine (VM) is abstraction between OS and hardware, and OS were installed on these virtual hardware.
%
2nd Generations Container is an abstraction between the servers and the OS, and it allows the OS to run multiple isolated user-space applications on the host OS.
%
As VM runs a complete OS on emulated hardware, it is too heavy to scale.
%
On the other hand, the Containers does not provide as strong security boundaries as VMs.
%
Between the spectrum of these two major types is the Light VM, which strikes balance between security and performance.